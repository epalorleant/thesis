\chapter{Introduction}
\label{chap:intro}

Event-based systems have gained importance in many application domains, such as management and control systems, large-scale data dissemination, monitoring applications, autonomic computing, etc. The components of an event-based system communicate by producing and consuming events, where an event is the notification that a happening of interest has occurred [VC02]. An event service mediates producers and consumers enabling loosely coupled communication among them. Producers publish events to the service, and consumers express their interest in receiving certain types of events by issuing event subscriptions. A subscription is seen as a continuous query that allows consumers to obtain event notifications [DGP07, BKK07]. The service is then responsible for matching received events with subscriptions and conveying event notifications to all interested consumers [MFP06]. Several academic research and industrial systems have tackled the problem of event composition. Techniques such as complex pattern detection [GJS92, GD94, CC96, PSB04], event correlation [Duo96, YB05], event aggregation [Luc02], event mining [AS95, GTB02] and stream processing [WDR06, DGP07, BKK07], have been used for composing events. In some cases event composition is done on event histories (e.g. event mining) and in other cases it is done dynamically as events are produced (e.g. event aggregation and stream processing). 

The extension of event models towards more flexible and Quality of Services (QoS) oriented event models requires an analysis of the semantics that should be given to the events, and of their associated processing strategies. This requires dissociating the modeling of events and the application design and, the proposal of methods that allow to define event types independently of the management issues (detection, production, notification). In addition, event composition varies according to the different application requirements, in other words, the semantics of composition operators and in the different production policies. Therefore, we need to adapt the event models to smart grid characteristics and requirements. We must define event types, composition operators with their associated semantics, and composition algorithms to produce QoS oriented complex events.
