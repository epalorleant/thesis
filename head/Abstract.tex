%\begingroup
%\let\cleardoublepage\clearpage


% English abstract
\cleardoublepage
\chapter*{Abstract}
%\markboth{Abstract}{Abstract}
\vspace{2cm}
\addstarredchapter{Abstract (English/Français)}
The abstract...
%\section*{Abstract}
%\addcontentsline{toc}{chapter}{Abstract (English/Français)} % adds an entry to the table of contents
% put your text here
% Nowadays, due to the increasing complexity in both the applications and the underlying hardware, it is difficult to understand what happens during the execution 
% of these applications. 
% Tracing techniques are commonly used to gather and provide information on application execution in the form of execution traces. 
% The execution traces, which are sequences of events, can be very large (easily millions of events), hard to understand and thus require specific analysis tools. 
% One critical case is the analysis of applications on embedded systems such as set-top boxes or smartphones, especially for understanding 
% bugs of multimedia applications. In this thesis we propose two novel analysis techniques adapted to multimedia applications 
%  on embedded systems. \\
% 
% The first method reduces size of trace given to the analysts. This method needs to group sets of related events together. 
% We propose an approach based on optimization and pattern mining techniques to automatically extract a set of subsequences from 
% an execution trace. Our experiments showed that the method scales on large amounts of 
% data and at the same time, highlighted the practical interest of this approach.\\
% 
% Our second contribution consists in proposing a diagnosis method based on the comparison of execution traces with reference traces. 
% This approach is implemented in {\em TED}, our TracE Diagnosis tool. 
% Experiments conducted on real-life use cases of multimedia application 
%  execution traces have validated that {\em TED} is scalable and brings 
% added value to traces analysis. We also show that the tool can be applied on reduced size traces in order to further improve scalability.

% The usage of execution traces as an analysis tool is increasingly common in many fields, particularly for debugging 
% multimedia applications in embedded systems. Execution traces are timestamped events sequences. 
% Their manual exploitation is almost impossible due to the large volume of information they contain. In this thesis, we propose two main techniques 
% that help capturing anomalies related to multimedia applications.\\
% 
% We first propose a method to extract a set of subsequences called {\em blocks}, from a sequence of events in order to abstract the trace and reduce 
% its size. These subsequences have the particularity to help the understanding of execution trace. Our experiments applied to multimedia applications traces 
% showed that the method scales on large amount of data. We also highlighted the practical interest of extracted blocks.\\
% Our second contribution consists in proposing a diagnosis method based on the comparison of execution  traces. This approach is 
% implemented in {\em TED}, our TracE Diagnosis tool. 
% Experiments and use cases conducted have validated that TED is scalable, has good precision and brings 
% added value to traces analysis. We later open a way towards an application of distances on reduced traces for improving the approach.
%\lipsum[1-2]
\vskip2cm
\textbf{Keywords: event stream processing, complex event processing, distributed systems, smart grids, Quality of service} .
%put your text here


%\vspace{5cm}

% French abstract
\begin{otherlanguage}{french}
\cleardoublepage
\chapter*{Résumé}
\vspace{2cm}
%\section*{Résumé}
%\markboth{Résumé}{Résumé}
% put your text here
% De nos jours, dû à la complexité croissante des applications et du matériel, il est difficile de comprendre ce qui se passe durant l'exécution 
% de ces applications. Les techniques de traçage sont communément utilisées pour collecter et fournir les informations sur l'application 
% sous forme de traces d'exécution. Les traces d'exécution, qui sont des séquences d'événements, peuvent être très volumineuses 
% (elles atteignent facilement des millions d'événements), 
% difficiles à comprendre et donc nécessitent des outils d'analyse spécifiques. Un cas critique est l'analyse d'applications pour systèmes embarqués 
% tels les décodeurs ou les smartphones, en particulier pour comprendre les bugs d'applications multimédias. Dans cette thèse, nous proposons 
% deux nouvelles techniques adaptées aux applications multimédia sur systèmes embarqués.\\
% 
% La première méthode réduit la taille de la trace donnée aux analystes. Cette méthode nécessite de regrouper un ensemble d'événements connexes. 
% Nous proposons une approche basée sur des techniques d'optimisation et de fouille de motifs afin d'extraire automatiquement un ensemble de 
% sous-séquences d'une trace. Nos expérimentations ont montré que cette méthode passe à l'échelle sur de gros volumes de données, et ont par la même 
% occasion mis en évidence l'intérêt pratique de cette approche.\\
% 
% La seconde contribution consiste en la mise en place d'une méthode de diagnostic basée sur la comparaison de traces d'exécution avec des traces 
% de référence. Cette méthode est implémentée dans {\em TED}, notre outil de diagnostic de traces. 
% Les expérimentations faites sur des cas d'utilisation concrets de traces d'exécution multimédia ont validé que {\em TED} passe à l'échelle et apporte 
% une plus-value à l'analyse de traces. Nous montrons aussi que l'outil peut être appliqué sur des traces de taille réduite 
% afin d'améliorer davantage le passage à l'échelle.

%\lipsum[1-2]
\vskip1.5cm
\textbf{Mots clefs}: event stream processing, complex event processing, distributed systems, smart grids, Quality of service. 
%put your text here
\end{otherlanguage}


%\endgroup			
%\vfill
