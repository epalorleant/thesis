%\usepackage[cam,a4,center,pdftex]{crop} %for markings
\usepackage[squaren]{SIunits} % provide \micro command
\usepackage{amsmath,amssymb}             % AMS Math
\usepackage[utf8]{inputenc}
\usepackage[T1]{fontenc}
\usepackage{lmodern}
\usepackage{units} % nicefrac
 \usepackage[english]{babel}
%\usepackage{csquotes}
%\usepackage[protrusion=true,expansion]{microtype}backend=biber,
%\usepackage[style=numeric-comp,citestyle=numeric-comp,bibstyle=numeric-short,sorting=none,natbib=true,backref=true,hyperref=true,autocite=superscript,maxnames=99,maxcitenames=2,isbn=false,url=false,eprint=false,firstinits=true,mcite,subentry]{biblatex}
%\usepackage[final]{microtype}
\usepackage[superscript]{cite}
\usepackage{microtype}
\usepackage{enumerate}
\usepackage{graphicx}
\usepackage{subfig}
%\usepackage[danish]{babel}
\usepackage{latexsym,amssymb,amsmath,verbatim}
\usepackage{array,booktabs}
\usepackage{placeins}
 \let\oldsection=\section % gemmer den gamle definition
 \renewcommand\section{\FloatBarrier\oldsection}
\usepackage{caption}
 \captionsetup{font=small,labelfont=bf}
%\usepackage{cite}
%\usepackage{hyperref}

%\DeclareFieldFormat{volcitevolume}{\bibstring{volume}\ppspace\RN{#1}}
%\AtEveryCitekey{\clearfield{institude}}
%\renewcommand{\cite}{\supercite}
%\renewcommand{\citep}{\supercite}
%\renewcommand{\citet}[2][]{\citeauthor{#2}\supercite{#2}\ifstrempty{#1}{}{, #1}}
% \DeclareFieldFormat{bibentrysetcount}{\newline(\mknumalph{#1})\addhighpenspace}
\usepackage{makeidx}
\usepackage{longtable}
\usepackage{calc}
%\usepackage{etoolbox}
\usepackage{leftidx}
\makeindex

\usepackage{booktabs}
\setlength{\heavyrulewidth}{1.2pt}
\usepackage{pdfpages}
\usepackage{geometry}
%\usepackage[left=3.15cm,right=2.55cm,top=1.8cm,bottom=0.89cm,includefoot,includehead,headheight=13.6pt]{geometry}
\usepackage{setspace}
\setstretch{1.1}
\usepackage[nottoc, notlof, notlot]{tocbibind}
\usepackage[nohints]{minitoc}
\setcounter{minitocdepth}{2}
\mtcindent=15pt
\renewcommand{\ptcfont}{\small\rm}
\renewcommand{\ptcCfont}{\small\bm}
\renewcommand{\ptcSfont}{\small\rm}
\renewcommand{\ptcSSfont}{\footnotesize\rm}
\renewcommand{\ptcSSSfont}{\footnotesize\rm}
% Use \minitoc where to put a table of contents
%\usepackage{cm-super}
\RequirePackage{fix-cm}

\usepackage{type1cm}
\usepackage{titlesec}

\usepackage{aecompl}
\usepackage{xargs}
% Glossary / list of abbreviations

\usepackage[intoc]{nomencl}
\renewcommand{\nomname}{List of Abbreviations}
%makeindex Thesis.nlo -o Thesis.nls

\makenomenclature
\usepackage{pgf}%,pgfarrows,pgfnodes,pgfautomata,pgfheaps,pgfshade}
\usepackage{tikz}
\usetikzlibrary{calc,decorations,decorations.pathmorphing,arrows,matrix,positioning,patterns}
\usepackage[version=3]{mhchem}% use \ce for chemistry
\usepackage{array}

% My pdf code


\usepackage{graphicx}
\DeclareGraphicsExtensions{.jpg}
\DeclareGraphicsExtensions{.pdf}

\graphicspath{{.}{images/}}

% Links in pdf
\usepackage{xcolor}
\selectcolormodel{cmyk}
\setcounter{secnumdepth}{3}
\setcounter{tocdepth}{2}

% Some useful commands and shortcut for maths:  partial derivative and stuff

\newcommand{\pd}[2]{\frac{\partial #1}{\partial #2}}
\def\abs{\operatorname{abs}}
\def\argmax{\operatornamewithlimits{arg\,max}}
\def\argmin{\operatornamewithlimits{arg\,min}}
\def\diag{\operatorname{Diag}}
\newcommand{\eqRef}[1]{(\ref{#1})}

\titlespacing*{\chapter}{0cm}{-1.cm}{-40pt}%pbk
\titleformat{\chapter}[display]{\Huge\filleft\scshape}{ \normalfont\bf\fontfamily{put}\fontseries{b}\fontsize{95pt}{0pt}\selectfont\thechapter}{20pt}{}[\titlerule\vspace{2ex}\filright\vspace{2ex}]
%
\titlespacing*{\part}{-10pt}{120pt}{-80pt}%pbk
\titleformat{\part}[frame]{\Huge\filcenter\scshape}{ \normalfont\bf\fontsize{70pt}{0pt}\selectfont \raisebox{1.9cm}{ Part\fontfamily{put}\fontseries{b}\fontsize{85pt}{0pt}\selectfont \hspace{.2em}\thepart} }{50pt}{}[\filright]

 \newcommand\boxedSection[3]{{%
%
     \begin{tikzpicture}[inner sep=#3,line width=1.0pt]
         \node[anchor=east,rectangle,draw] at (0,0) (counter) {\textbf{#2}};
             \draw (counter.south west)  ++(.0pt,.5pt)-- ++($(\linewidth,0) - (2.5pt,0)$);
\node [right of=counter,anchor=west]{#1};
     \end{tikzpicture}
 }}
 \newcommand\boxedSectionB[3]{{%
%
     \begin{tikzpicture}[inner sep=#3,line width=1pt]
         \node[anchor=east,rectangle,draw,fill=black] at (0,0) (counter) {\color{white}\textbf{#2}};
             \draw (counter.south west) ++(.0pt,.5pt)-- ++($(\linewidth,0) - (2.5pt,0)$);
\node [right of=counter,anchor=west]{#1};
     \end{tikzpicture}
 }}
\newcommand\boxedsection[1]{\boxedSectionB{#1}{\thesection}{2mm}}
\newcommand\boxedsubsection[1]{\boxedSection{#1}{\thesubsection}{1.7mm}}
\newcommand\boxedsubsubsection[1]{\boxedSection{#1}{\thesubsubsection}{1.5mm}}
 \titleformat{\section}[hang]%
     {\usekomafont{section}}%
     {}%
     {.0em}%
     {\filright\boxedsection}%

 \titleformat{\subsection}[hang]%
     {\usekomafont{subsection}}%
     {}%
     {.0em}%
     {\filright\boxedsubsection}%
 \titleformat{\subsubsection}[hang]%
     {\usekomafont{subsubsection}}%
     {}%
     {.0em}%
     {\filright\boxedsubsubsection}%




\usepackage{rotating}
\usepackage{fancyhdr}                    % Fancy Header and Footer

% \usepackage{txfonts}                     % Public Times New Roman text & math font

%%% Fancy Header %%%%%%%%%%%%%%%%%%%%%%%%%%%%%%%%%%%%%%%%%%%%%%%%%%%%%%%%%%%%%%%%%%
% Fancy Header Style Options

\pagestyle{fancy}                       % Sets fancy header and footer
\fancyfoot{}                            % Delete current footer settings

%\renewcommand{\chaptermark}[1]{         % Lower Case Chapter marker style
%  \markboth{\chaptername\ \thechapter.\ #1}}{}} %

%\renewcommand{\sectionmark}[1]{         % Lower case Section marker style
%  \markright{\thesection.\ #1}}         %

\fancyhead[LE,RO]{\bfseries\thepage}    % Page number (boldface) in left on even
% pages and right on odd pages
\renewcommand{\chaptermark}[1]{\markboth{\MakeUppercase{\thechapter.\ #1}}{}}
\fancyhead[RE]{\bfseries\nouppercase{\leftmark}}      % Chapter in the right on even pages
\fancyhead[LO]{\bfseries\nouppercase{\rightmark}}     % Section in the left on odd pages

\let\headruleORIG\headrule
\renewcommand{\headrule}{\color{black} \headruleORIG}
\renewcommand{\headrulewidth}{1.0pt}
\usepackage{colortbl}
\arrayrulecolor{black}

\fancypagestyle{plain}{
  \fancyhead{}
  \fancyfoot{}
  \renewcommand{\headrulewidth}{0pt}
}

\usepackage{algorithm}
\usepackage[noend]{algorithmic}


%%% Clear Header %%%%%%%%%%%%%%%%%%%%%%%%%%%%%%%%%%%%%%%%%%%%%%%%%%%%%%%%%%%%%%%%%%
% This redefinition of the \cleardoublepage command provides
% for a special pagestyle for the "extra" pages which are generated
% to ensure that the chapter opener is on a recto page.
% The pagestyle is "chapterverso"; for many publishers, this should be
% identical to "empty", so that's the default.

% Clear Header Style on the Last Empty Odd pages
\makeatletter

\def\cleardoublepage{\clearpage\if@twoside \ifodd\c@page\else%
  \hbox{}%
  \thispagestyle{empty}%              % Empty header styles
  \newpage%
  \if@twocolumn\hbox{}\newpage\fi\fi\fi}

\makeatother

%%%%%%%%%%%%%%%%%%%%%%%%%%%%%%%%%%%%%%%%%%%%%%%%%%%%%%%%%%%%%%%%%%%%%%%%%%%%%%%
% Prints your review date and 'Draft Version' (From Josullvn, CS, CMU)
\newcommand{\reviewtimetoday}[2]{\special{!userdict begin
    /bop-hook{gsave 20 710 translate 45 rotate 0.8 setgray
      /Times-Roman findfont 12 scalefont setfont 0 0   moveto (#1) show
      0 -12 moveto (#2) show grestore}def end}}
% You can turn on or off this option.
%\reviewtimetoday{\today}{Draft Version}
%%%%%%%%%%%%%%%%%%%%%%%%%%%%%%%%%%%%%%%%%%%%%%%%%%%%%%%%%%%%%%%%%%%%%%%%%%%%%%%

\usepackage{multirow}
\usepackage{slashbox}

\renewcommand{\epsilon}{\varepsilon}

% centered page environment

\newenvironment{vcentrepage}
{\newpage\vspace*{\fill}\thispagestyle{empty}\renewcommand{\headrulewidth}{0pt}}
{\vspace*{\fill}}


\renewcommand{\figurename}{Figure~}
%\usepackage[small,bf]{caption}

\addtokomafont{caption}{\small}
\setkomafont{captionlabel}{\small\bfseries}

\setcapindent{0em}
\usepackage{ifpdf}

\usepackage[b5paper,hyperindex=true,plainpages=false]{hyperref}


\renewcommand{\thefootnote}{\roman{footnote}}
\def\sectionautorefname{Section}
\def\subsectionautorefname{Section}
\def\subsubsectionautorefname{Section}
\usepackage{pdflscape}

\newenvironment{mysidewaystable}[1][htp]{\begin{sidewaystable}[#1]}{\end{sidewaystable}}

\renewcommand{\topfraction}{0.7}	% max fraction of floats at top
\renewcommand{\bottomfraction}{0.75}	% max fraction of floats at bottom
%   Parametres for TEXT pages (not float pages):
\setcounter{topnumber}{2}
\setcounter{bottomnumber}{1}
\setcounter{totalnumber}{3}     % 2 may work better
 \setcounter{dbltopnumber}{2}    % for 2-column pages
\renewcommand{\textfraction}{0.05}	% allow minimal text w. figs
%   Parametres for FLOAT pages (not text pages):
\renewcommand{\floatpagefraction}{0.6}	% require fuller float pages
% N.B.: floatpagefraction MUST be less than topfraction !!
\usepackage[title,toc]{appendix}
\renewcommand{\thechapter}{\thepart.\arabic{chapter}}

